\section*{1.2}

\subsection*{الف}
برای اثبات بستگی زبان‌های مستقل از متن دربرابر عمل الحاق، دو گرامر زیر را در نظر می‌گیریم:
\[
G_1 = (V_1, \Sigma_1, R_1, S_1) \quad G_2 = (V_2, \Sigma_2, R_2, S_2)
\]
گرامر الحاق این دو به شکل زیر خواهد بود:
\[
G_T = (V_1 \cup V_2 \cup \{S_0\}, \Sigma_1 \cup \Sigma_2, R_1 \cup R_2 \cup \{S_0 \to S_1 S_2\}, S_0)
\]
که در آن یک نشان‌دهندهٔ جدید، یعنی $S_0$، تعریف شده است و یک ارتباط جدید، $S_0 \to S_1 S_2$، برقرار شده است. این گرامر نشان‌دهندهٔ عمل الحاق است، بنابراین زبان‌های مستقل از متن دربرابر این عمل بسته هستند.

\subsection*{ب}
حالا یک گرامر دلخواه، $G_1 = (V_1, \Sigma_1, R_1, S_1)$، را در نظر بگیرید. برای عملگر $*$، گرامری را می‌توان به شکل زیر در نظر گرفت:
\[
G_{*} = (V_1 \cup \{S_{*}\}, \Sigma_1, R_1 \cup \{S_{*} \to S_1 S_{*} \,|\,\epsilon\}, S_{*})
\]
مانند بخش الف، یک نشان‌دهندهٔ جدید، $S_{*}$، در نظر گرفته شده است و با افزودن قاعده $S_{*} \to S_1 S_{*} \,|\,\epsilon$ به مجموعه قواعد، می‌توان عملگر $*$ را پیاده ساخت. پس زبان‌های مستقل از متن نسبت به این عملگر نیز بسته خواهند بود.

\section*{2.2}

\subsection*{الف}
با توجه به زبان مستقل از متن ارائه شده، زبان زیر را در نظر بگیرید:
\[
L = \{a^m b^k c^n \,|\, m,n \geq 0 \,and\, (m = k + n \,or\, k = m + n \,or\, n = k + m)\}    
\]
با توجه به بستگی زبان‌های مستقل از متن دربرابر عمل اجتماع، می‌توان زبان بالا را به سه زبان مجزا تقسیم کرد و با اثبات مستقل بودن هر یک از آن‌ها، در نهایت به مستقل بودن زبان $L$ می‌رسیم. بنابراین، زبان‌های زیر را در نظر بگیرید:
\begin{align*}
	L_1 &= \{a^m b^k c^n \,|\, m = k + n\} \\
	L_2 &= \{a^m b^k c^n \,|\, n = k + m\} \\
	L_3 &= \{a^m b^k c^n \,|\, k = m + n\}
\end{align*}
در ابتدا برای زبان $L_1$، گرامر مستقل از متن زیر را می‌نویسیم:
\begin{align*}
	S &\to a S c \,|\, X \\
	X &\to a X b \,|\, \epsilon
\end{align*}
که در آن جمع تعداد تکرار حروف $c$ و $b$ با $a$ برابر خواهد بود، چرا که برای هر $a$، یک $c$ یا یک $b$ وجود دارد. بنابراین، گرامر زبان $L_1$ مستقل از متن است.

به همین ترتیب، برای زبان $L_2$، گرامر مستقل از متن زیر را در نظر بگیرید:
\begin{align*}
	S &\to a S c \,|\, X \\
	X &\to b X c \,|\, \epsilon
\end{align*}
که در آن جمع تعداد تکرار $a$ و $b$ برابر با $c$ خواهد بود. بنابراین، این زبان نیز یک زبان مستقل از متن است.

در نهایت، برای زبان $L_3$، می‌توان دو زبان مستقل از متن زیر را الحاق کرد:
\[
L_4 = (a^m b^k \,|\, m = k) \quad
L_5 = (b^k c^n \,|\, k = n)     
\]
که می‌دانیم این دو زبان مستقل از متن هستند، پس الحاق آنها نیز مستقل از متن بوده و در نیتجه زبان $L$ مستقل از متن خواهد بود.

برای بخش بعدی، زبان زیر را در نظر بگیرید:
\[
L = \{a^{m_1} b^{k_1} c^{n_1} \dots a^{m_i} b^{k_i} c^{n_i} \,|\, i \geq 0 \,and\, \forall j \leq i \, m_j, n_j \geq 0 \,and\, k_j = 3m_j + 4n_j\}    
\]
از آنجایی که زبان‌های مستقل از متن نسبت به عمل $*$ بسته هستند، می‌توان زبان $L$ را به این صورت تقلیل داد:
\[
L = (L_1)^{*} \implies L_1 = (a^m b^k c^n \,|\, k = 3m + 4n)    
\]
بنابراین کافی است نشان دهیم زبان $L_1$ مستقل از متن است. برای اینکار، دو زبان زیر را با گرامر متناظر با آنها در نظر می‌گیریم:
\[
L_2 = (a^m b^k \,|\, k = 3m) \quad S \to aSbbb \,|\, \epsilon    
\]
\[
L_3= (b^k c^n \,|\, k = 4n) \quad S \to bbbbSc \,|\, \epsilon    
\]
با الحاق این دو زبان، $L_1$ ساخته می‌شود. از آنجایی که هرکدام از آنها مستقل از متن هستند و عملگر الحاق نیز نسبت به این کار بسته است، در نهایت زبان $L$ یک زبان مستقل از متن خواهد بود.
