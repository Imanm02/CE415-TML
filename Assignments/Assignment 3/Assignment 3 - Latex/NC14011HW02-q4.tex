\section*{1.4}

\subsection*{الف}
به زبان $L_1$ بیان شده به صورت زیر توجه می‌کنیم:
\[
L_1 = \{w\#x\#y\#z \,|\, w,x,y,z \in {a,b}^{*} \,and\, |w| = |z|,\, |x| = |y|\}    
\]
برای گرامر این زبان خواهیم داشت.
\begin{align*}
	S &\to C S C \,|\, A \\
	A &\to \# B \# \\
	B &\to C B C \,|\, \# \\
	C &\to a \,|\, b
\end{align*}
در اینجا می‌دانیم عبارت با حروف
$a$
و
$b$
شروع و خاتمه پیدا می‌کند. همچین در بین علامت‌های
$\#$
طوری گرامر را قرار داده‌ایم که اندازه دو دنباله وسطی با یکدیگر و دو دنباله ابتدایی و انتهایی با هم برابر باشند.

\subsection*{ب}
در اینجا داریم.
\[
L_2 = \{w\#x\#y\#z \,|\, w,x,y,z \in {a,b}^{*} \,and\, |w| = |y|,\, |x| = |z|\}  
\]
نشان می‌دهیم برعکس قسمت بالا این زبان مستقل از متن نیست. طبق لم تزریق بدست می‌آوریم که:
\[
\implies w\#x\#y\#z = (a|b)^p \# (a|b)^p \# (a|b)^p \# (a|b)^p = uvxyz 
\]
در اینجا دو بخش میانی یعنی
$y$
و
$v$
نمی‌توانند دارای علامت‌
$\#$
باشند زیراکه با تزریق این قسمت بیش از سه هشتگ خواهیم داشت. پس این دو قسمت یا هر دو داخل یکی از زیر رشته‌های
$w$، $x$، $y$ و $z$
قرار داشته و یا در دو زیررشته کناری قرار دارند.

اگر هر دو در زیررشته‌های کناری باشند، از آنجایی که اندازه دو دنباله
$x$
و
$z$
و دو دیگر دنباله باید برابر باشند، پامپ شدن این قسمت تساوی به هم خورده و رشته جدید تولید شده خارج از زبان
$L_2$
قرار می‌گیرد.

اگر هر دو در یک زیررشته نیز باشند با اعمال تزریق تناسب زیررشته‌ها به هم خورده پس می‌توان نتیجه گرفت که براساس لم تزریق رشته‌های تولید شده خارج از زبان قرار گرفته و این زبان یک زبان مستقل از متن نخواهد بود.

\section*{2.4}

\subsection*{الف}
\[
L_1 = \{a^{2n} b^{3n} c^n \,|\, n \geq 0\}    
\]
با توجه به لم تزریق رشته زیر را درنظر می‌گیریم.
\[
\implies a^{2p} b^{3p} c^p = uvxyz    
\]
با توجه به رشته، هرکدام از رشته‌های
$v$
و
$y$
فقط شامل یک حرف خواهند بود زیرا در غیر این صورت با افزایش آنها تناسب رشته به هم می‌ریزد. حال سه حالت داریم.
\begin{itemize}
	\item
	اگر هر دو یک حرف یکسان باشند، با انجام تزریق تعداد آن حرف بیشتر از مابقی شده و تناسب رشته به هم می‌ریزد.
	
	\item
	اگر
	$v$
	برای حرف
	$a$
	و
	$y$
	برای حرف
	$b$
	باشد. با پامپ کردن این دو مقدار نسبت حروف به حرف
	$c$
	به هم خورده و رشته‌ای خارج از زبان خروجی می‌دهد.
	
	\item
	رشته
	$v$
	برای حرف
	$b$
	و
	$y$
	برای حرف
	$c$
	باشد. همانند قسمت قبل تناسب رشته نسبت به حرف
	$a$
	به هم می‌ریزد.
\end{itemize}
در نتیجه در تمامی حالات رشته‌ای خارج از زبان
$L_1$
تولید شده و این زبان مستقل از متن نیست.

\subsection*{ب}
\[
L_2 = \{a^{(n - 1)(n + 1)} \,|\, n > 0\}    
\]
همانند قسمت قبل داریم.
\[
\implies a^{(p - 1)(p + 1)} = a^{p^2 - 1} = u v x y z    
\]
در اینجا دو رشته
$v$
و
$y$
تنها از حرف
$a$
تشکیل شده‌اند و برای آنها خواهیم داشت.
\[
|u y| = t, \quad 0 < t \leq p    
\]
\[
\implies |u v^2 x y^2 z| = p^2 - 1 + t    
\]
\[
\implies p^2 - 1 < p^2 - 1 + t  \leq p^2 + p - 1   
\]
\[
\implies p^2 + p - 1 < p^2 + 2p
\]
این نشان دهنده این است که طول رشته
$u v^2 x y^2 z$
بین دو مقدار
$p^2 - 1$
و
$p^2 + 2p$
خواهد بود. در صورتی که در زبان
$L_2$
این رشته با این طول پذیرفته نمی‌شود چون نمی‌تواند طولی بین این دو مقدار برای رشته تولید کند.
\[
n = p \implies a^{(p - 1)(p + 1)} = a^{p^2 - 1}
\]
\[
n = p + 1 \implies a^{(p)(p + 2)} = a^{p^2 + 2p}
\]
بنابراین با استفاده از لم تزریق رشته‌ای تولید می‌شود که خارج از زبان است پس این زبان تمی‌تواند مستقل از متن باشد.

\subsection*{ج}
\[
L_3 = \{a^n b^n c^i \,|\, n \leq i \leq 2n\}    
\]
در اینجا داریم.
\[
\implies a^p b^p c^{2p} = u v x y z    
\]
در اینجا دو رشته میانی
$v$
و
$y$
فقط می‌توانند شامل یک حرف باشند زیرا در غیر این صورت نظم رشته را از دست خواهیم داد.

حال اگر این دو رشته شامل یک حرف باشند با تکرار آنها یکی از عناصر
$a$، $b$ و $c$
بیشتر از بقیه شده و رشته‌ای خارج از زبان
$L_3$
تولید می‌شود.

اگر
$v$
شامل
$b$
باشد با تکرار آنها تعداد حالت‌ها از
$a$
بیشتر شده و این حالت نیز قابل قبول نیست.

در حالت
$v$
شامل
$a$
باشد و 
$y$
متشکل از
$b$
باشد بدست می‌آوریم:
\[
\implies |v| = |y| = t \quad 1 \leq t \leq p    
\]
اگر دو رشته مدنظر را صفر بار تزریق کنیم خواهیم داشت.
\[
\implies a^{p - t} b^{p - t} c^{2p} = u x z \implies 2p \leq 2p - 2t    
\]
در نتیجه با توجه به شرط زبان
$L_3$
این رشته خارج از این زبان بوده و این یک زبان غیرمستقل از متن خواهد بود.