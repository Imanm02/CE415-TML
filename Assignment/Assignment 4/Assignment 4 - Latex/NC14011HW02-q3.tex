\subsection*{1.1.3}

یک ماشین تصمیم‌گیرنده را می‌توان با یک رشته بایتی که وضعیت‌ها، الفبا، تابع انتقال، وضعیت اولیه و مجموعه وضعیت‌های پایانی را توصیف می‌کند، نشان داد. برای هر ماشین تصمیم‌گیرنده مختلف، یک رشته بایتی متمایز وجود دارد که آن را توصیف می‌کند، و برعکس، برای هر رشته بایتی مختلف، یک ماشین تصمیم‌گیرنده متمایز وجود دارد.

با این حال، تعداد رشته‌های بایتی متمایز شمارا است، چرا که هر رشته بایتی می‌تواند هر طول طبیعی را داشته باشد و برای هر بایت در رشته، 256 انتخاب مختلف وجود دارد (از 0 تا 255). بنابراین، تعداد ماشین‌های تصمیم‌گیرنده نیز شمارا است.

\subsection*{2.1.3}

\lr{L = {⟨M⟩ : M is a Turing decider such that ⟨M⟩ $\notin$ L(M)}}

 زبانی است که شامل تمام ماشین‌های تورینگ است که خودشان را نمی‌پذیرند. این زبان برابر با مشهورترین مثال از یک زبان ناتصمیم‌پذیر است، مشهور به مسئله رازنیک هالینگ.

در این زمینه، فرض کنید که M یک ماشین تورینگ است که L را تصمیم می‌گیرد. اکنون دو حالت ممکن است:

\lr{⟨M⟩ $\in$ L. }


این بدین معناست که M خود را نمی‌پذیرد، که با فرض اینکه M یک تصمیم‌گیر برای L است تناقض دارد.

\lr{⟨M⟩ $\notin$ L. }


این بدین معناست که M خود را می‌پذیرد، که باز هم با فرض اینکه M یک تصمیم‌گیر برای L است تناقض دارد.

بنابراین، هیچ ماشین تورینگی وجود ندارد که بتواند L را تصمیم بگیرد، پس L ناتصمیم‌پذیر است.

\subsection*{2.3}

یکی از مهم‌ترین زبان‌هایی که این خاصیت را دارد، مسئله‌ی رازنیک هالینگ 
\lr{(Halting Problem)}
 است. این مسئله به شکل زیر تعریف می‌شود:

\lr{H = {⟨M,w⟩ : M is a Turing machine that halts on input w}}

این زبان ناتصمیم‌پذیر است، اما هر زبان تصمیم‌پذیر دیگر به آن کاهش می‌یابد. برای نشان دادن این موضوع، فرض کنید که A یک زبان تورینگ-تصمیم‌پذیر دلخواه باشد. پس ماشین تورینگ M وجود دارد که A را تصمیم می‌گیرد. اکنون می‌توانیم تابع کاهش f را به شکل زیر تعریف کنیم:

\lr{f(⟨x⟩) = ⟨M,x⟩}

این تابع برای هر ورودی x ، یک زوج ماشین تورینگ و ورودی به ما می‌دهد. اکنون، x در A است اگر و تنها اگر M برای ورودی x متوقف شود. بنابراین، x در A است اگر و تنها اگر f(x) در H باشد، که این نشان‌دهنده‌ی این است که A به H کاهش می‌یابد.

اگر زبان A به زبان L کاهش یابد و L تورینگ-تصمیم‌پذیر باشد، آنگاه A نیز تورینگ-تصمیم‌پذیر است. این را می‌توان با استفاده از تابع کاهش f نشان داد: اگر M یک ماشین تورینگ باشد که L را تصمیم می‌گیرد، پس ما می‌توانیم ماشین تورینگ M' را تعریف کنیم که برای هر ورودی x ، ابتدا f(x) را محاسبه کرده و سپس M را بر روی نتیجه اجرا کند. اگر M برای f(x) متوقف شود، پس
$M' x$
  را می‌پذیرد. در غیر این صورت،
$M' x$
را رد می‌کند. بنابراین، اگر A به L کاهش یابد و L تورینگ-تصمیم‌پذیر باشد، آنگاه A نیز تورینگ-تصمیم‌پذیر است.


\subsection*{3.3}

با رسیدن به تناقض بدست می‌آوریم.

فرض کنید T تشخیص‌پذیر است. اگر اینطور باشد، می‌توانیم یک ماشین تورینگ D وجود داشته باشد که برای یک ورودی w، اگر w توصیف‌کننده یک ماشین تورینگ طلایی باشد، D w را بپذیرد. در غیر این صورت، D ، w را رد می‌کند.

حالا بیایید ماشین تورینگ D' را تعریف کنیم به این شکل که برای یک ورودی w، اگر D w را بپذیرد، D' w را رد می‌کند و برعکس.

اکنون به نظر می‌رسد D' یک ماشین تورینگ طلایی باشد، چون D' یک ورودی w را پذیرفته یا رد می‌کند بر اساس اینکه آیا D w را پذیرفته یا رد می‌کند. با این حال، اگر D' یک ماشین تورینگ طلایی باشد، آنگاه D باید
\lr{⟨D'⟩}
را بپذیرد، که باعث می‌شود
\lr{D' ⟨D'⟩}
را رد کند. این با فرض اینکه D' یک ماشین تورینگ طلایی است، تناقض است. بنابراین، T نمی‌تواند تشخیص‌پذیر باشد.
